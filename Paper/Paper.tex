\documentclass[12pt]{article}

\usepackage{amsmath}
\usepackage{amssymb}
\usepackage{amsthm}
\usepackage{graphicx}
\usepackage{float}

\newcommand{\R}{\mathbb{R}}
\newcommand{\V}{\mathbb{V}}
\newcommand{\prl}{\parallel}
\newcommand{\prp}{\perp}

\title{Billiard Ball Physics}
\author{Spencer T. Parkin}

\begin{document}
\maketitle

Letting $m_1,m_2\in\R$ be the two masses of two billiard balls, and $v_1,v_2\in\V$ be their respective velocities,
suppose these balls are in collision with one another at a contact point with unit-length contact normal $n\in\V$.
We will let $x_1,x_2\in\V$ denote, respectively, the new velocities of these billiard balls after the collision.
That said, following the advice of my friend Paul Johnston, two quantities are preserved: the total momentum
of the system, and the total kinetic energy.  According to my physics text, the first of these is always conserved, and
the second is conserved in the case of what's called a perfectly elastic collision.  For a brief moment, the kinetic energy
of the collision is converted to potential energy, but then all of that potential energy is converted back into kinetic energy,
none of it getting lost to heat or sound or anything else.

Letting
\begin{align*}
P(a,b)&=m_1a+m_2b,\\
K(a,b)&=\frac{1}{2}m_1a^2+\frac{1}{2}m_2b^2,
\end{align*}
this means that
\begin{align*}
P(v_1,v_2)&=P(x_1,x_2),\\
K(v_1,v_2)&=K(x_1,x_2).
\end{align*}
Now for any $a\in\V$, let $a^{\prl}=(a\cdot n)n$ and $a^{\prp}=a-a^{\prl}$.  Doing so, it is not hard to show that
\begin{align*}
P(x_1,x_2) &= P(x_1^{\prl},x_2^{\prl})+P(x_1^{\prp},x_2^{\prp}), \\
K(x_1,x_2) &= K(x_1^{\prl},x_2^{\prl})+K(x_1^{\prp},x_2^{\prp}),
\end{align*}
and similarly for $v_1$ and $v_2$.\footnote{This means that $P$ and $K$ are bilinear functions.}
Therefore, after making a physics-based argument that we must have $v_1^{\prp}=x_1^{\prp}$ and $v_2^{\prp}=x_2^{\prp}$,
our preservation equations become
\begin{align*}
P(v_1^{\prl},v_2^{\prl})&=P(x_1^{\prl},x_2^{\prl}),\\
K(v_1^{\prl},v_2^{\prl})&=K(x_1^{\prl},x_2^{\prl}).
\end{align*}
We will now write $P^{\prl}=P(v_1^{\prl},v_2^{\prl})$ and $K^{\prl}=K(v_1^{\prl},v_2^{\prl})$ for short.
Solving for $x_2^{\prl}$ in the momentum preservation equation, and then plugging that into
the kinetic energy preservation equation, we find that
\begin{equation*}
0 = \frac{1}{2}m_1\left(1+\frac{m_1}{m_2}\right)(x_1^{\prl})^2 - \frac{m_1}{m_2}P^{\prl}\cdot x_1^{\prl} + \frac{(P^{\prl})^2}{2m_2} - K^{\prl}.
\end{equation*}
Though not immediately obvious, this is a quadratic equation in the variable $x_1\cdot n$.  To see that, we write
\begin{equation*}
0 = \frac{1}{2}m_1\left(1+\frac{m_1}{m_2}\right)(x_1\cdot n)^2 - \frac{m_1}{m_2}(P\cdot n)(x_1\cdot n) + \frac{(P\cdot n)^2}{2m_2} - K^{\prl}.
\end{equation*}
If we find a solution for $(x_1\cdot n)$, we can then find $x_1$ as
\begin{equation*}
x_1=v_1^{\prp}+(x_1\cdot n)n.
\end{equation*}
By symmetry of the preservation equations, a similar solution is found for $x_2$.  (Just replace subscript 1 with 2 and vice-versa.)

Back to our quadratic equation, if we examine the descriminant $D$, it becomes
\begin{equation*}
D = 2m_1\left(1+\frac{m_1}{m_2}\right)K^{\prl}-\frac{m_1}{m_2}(P\cdot n)^2.
\end{equation*}
If we continue to expand this, we eventually find that
\begin{equation*}
D = m_1^2\left((v_1\cdot n)^2 + (v_2\cdot n)^2 - 2(v_1\cdot n)(v_2\cdot n)\right).
\end{equation*}
Interestingly, it is clear that this factors as
\begin{equation*}
D = m_1^2\left((v_2-v_1)\cdot n\right)^2.
\end{equation*}
Immediately we see that $D\geq 0$, and so our quadratic equation always has at least one real zero.
Applying our quadratic formula, we arrive at
\begin{equation*}
x_1\cdot n = \frac{P\cdot n\pm m_2|(v_2-v_1)\cdot n|}{m_1+m_2}.
\end{equation*}
Putting it all together now, we finally have
\begin{equation*}
x_1 = v_1 + \left(\frac{P\cdot n\pm m_2(v_2-v_1)\cdot n}{m_1+m_2}-v_1\cdot n\right)n.
\end{equation*}
Notice that we could drop the absolute value because of the plus or minus.  Now, if we examine the
negative case, we find that $x_1$ no longer becomes dependent upon $v_2$, which doesn't make intuitive sense.
So the case we care about must be the positive case.  In that case, we get
\begin{equation*}
x_1 = v_1 + \left(\frac{2m_2}{m_1+m_2}(v_2\cdot n)+\left(\frac{m_1-m_2}{m_1+m_2}-1\right)(v_1\cdot n)\right)n.
\end{equation*}
Using this equation, we can now resolve collisions between two billiard balls.  Notice that it didn't matter which direction $n$ pointed, as long as it was a normal to the contact surface.

Let's check some intuition here.
Suppose the first ball was at rest when the second ball came and hit it.  Further, let's suppose the two balls
had the same mass.  In math terms, this means that $m_1=m_2$ and $v_1=0$.  With these, we find
that $x_1=(v_2\cdot n)n=v_2^{\prl}$, which makes sense.  If the first ball was hit head-on, then $v_2=v_2^{\prl}$,
and we simply have $x_1=v_2$, showing that the first ball inherited all of the velocity of the second.  So what
happened to the velocity of the second ball under this same situation?  Using our formulate (replacing 1 with 2 and vice-versa),
we find that $x_2=v_2-v_2^{\prl}$.  Again, if we hit it head-on, then $v_2=v_2^{\prl}$, and we get $x_2=0$,
which makes sense.  The second ball stops while the first ball inherits all of the second ball's velocity.

Let's now consider a billiard ball colliding with the edge of the pool table.  In this case, we take the limit
as $m_2$ goes to $\infty$, and let $v_2=0$, assuming the first billiard ball is the ball in question here.
Using our formulate for $x_1$, we get
\begin{equation*}
x_1 = v_1 - 2(v_1\cdot n)n = -nv_1n,
\end{equation*}
which we clearly recognize as a reflection in geometric algebra.  So the ball simply bounces off the edge of the table.

\end{document}